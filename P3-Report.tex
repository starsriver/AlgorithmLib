\documentclass[a4paper]{article}
\usepackage{booktabs}
\usepackage{geometry}
\geometry{
  top=1in,
  inner=1in,
  outer=1in,
  bottom=1in,
  headheight=3ex,
  headsep=2ex
}
\usepackage{amssymb,amsmath}
\usepackage{fontspec}
\usepackage[CJKbookmarks, colorlinks, bookmarksnumbered=true,pdfstartview=FitH,linkcolor=black,citecolor=black]{hyperref}
\usepackage{xeCJK}
\usepackage{xltxtra,xunicode}
\usepackage{listings}
\usepackage{xcolor}
\usepackage{array}

\lstset{basicstyle=\footnotesize\ttfamily,        % size of fonts used for the code
        columns=fullflexible,
        numbers=left,
        numberstyle=\tiny,
        keywordstyle=\color{blue},
        stringstyle=\color[rgb]{0,0.6,0},
        commentstyle=\color[cmyk]{1,0,1,0},
        frame=shadowbox,
        escapeinside=``,
        breaklines=true,
        extendedchars=false,
        xleftmargin=2em,xrightmargin=2em, aboveskip=1em,
        tabsize=4, %tab size
        showspaces=false %no space
       }

\newcommand{\tightlist}{
  \setlength{\itemsep}{0pt}\setlength{\parskip}{0pt}}

% font
\setCJKmainfont[AutoFakeBold]{文泉驿微米黑}
\setmainfont[AutoFakeBold]{Segoe UI}
\setromanfont[AutoFakeBold]{Segoe UI}
\setmonofont[Mapping={}]{Monaco}
\linespread{1.2}\selectfont
\XeTeXlinebreaklocale "zh" 

\begin{document}
\title{\Huge 算法导论课程\\ 第三次上机实验报告}
\author { \vspace{12cm} \\ \LARGE 班级:  1413014  \\ \LARGE 姓名:  乔新文   \\ \LARGE 学号:14130140393} 
\date{ \vspace{4cm} 2017.4.26}

\maketitle
\clearpage

\tableofcontents

\clearpage

\section{综述}

本文档将阐述《算法导论》第二次上机实验代码的详细设计及实现。\\

本次上机实验所用代码均为F\#代码,主要算法和辅助定义均包含在SRAlgorithmLib命名空间下AlgorithmLib3.fs所定义的AlgorithmLib3模块中。\\

算法测试驱动部分在Test.fs中Test模块下的P3函数中,整个程序的入口在Program.fs中。\\

本次实验所有代码均运行在.Net Core上,P1.fsproj为项目配置文件,可在安装有.Net Core的环境中在项目文件夹下使用dotnet run命令运行。\\

\section{题目一}

\subsection{题目}

0-1 knapsack problem.

\subsection{实现思路}

由于贪心算法对0-1背包问题无效,故采用自顶向下的递归的动态规划方法求解。\\
当背包剩余容量小于等于0时,显然递归函数返回0。\\
当剩余一个物品时,如果物品重量大于背包剩余容量,那么函数返回0,否则返回该物品价值。\\
当剩余多个物品时,如果第一个物品重量大于背包剩余容量,排除该物品。否则考虑放入该物品和不放入该物品两种情况,取价值较大的情况。\\
那么递归方程如下:
\[ks(c,w,v)=
    \left\{
        \begin{array}{ll}
            0 & ,(c \le 0 )\ or \ (|w|=1 \ and \ w.Head > c) \\
            v.Head & ,|w|=1 \ and \ w.Head \le c \\
            ks(c,w.Tail,v.Tail) & , w.Head \le c \\
            max(ks(c,w.Tail,v.Tail),ks(c-w.Head,w.Tail,v.Tail)+v.Head) & , else
        \end{array}
    \right.
\]

\subsection{实现代码}

自顶向下的递归方法如下

\begin{lstlisting}[language=ML]
    let rec ks c w (v:int list)=
        if c <= 0 then 0
        elif List.length w = 1 then
            if w.Head > c then 0
            else v.Head
        else
            let w' = w.Tail
            let v' = v.Tail
            if w.Head > c then 
                ks c w' v'
            else
                max (ks c w' v') ((ks (c - w.Head) w' v') + v.Head)
\end{lstlisting}

\subsection{算法测试}

\begin{itemize}
\item
    测试样例:
    \begin{itemize}
    \item
        w = [50;30;45;25;5] \\
        v = [200;180;225;200;50]
    \end{itemize}
\item
    样例输出
    \begin{itemize}
    \item
        605
    \end{itemize}
\end{itemize}


\section{题目二}

\subsection{题目}

Fractional knapsack problem.

\subsection{实现思路}

使用贪心策略,即尽可能的装入更多单位重量价值更高的东西,直到装满或没有东西剩余为止。

\subsection{实现代码}

递归的应用贪心策略,要求传入的vWith1W是从大到小有序的,w与vWith1W严格对应。

\begin{lstlisting}[language=ML]
    let rec ksfWithSorted c w (vWith1W : float list) =
        if c <= 0.0 || List.length w <= 0 then 0.0
        else
            if c >= w.Head then
                w.Head * vWith1W.Head + (ksfWithSorted (c - w.Head) w.Tail vWith1W.Tail)
            else
                c * vWith1W.Head
\end{lstlisting}

计算每件物品的单位重量价值,并与其重量序列严格对应进行排序,即对传入的重量列表,价值列表进行预处理后再调用ksfWithSorted。\\

转换c和w为float类型,便于后期计算。

\begin{lstlisting}[language=ML]
        let c' = float c
        let wf = List.map (fun item -> (item |> float)) w
\end{lstlisting}

计算每个物品单位重量的价值。
\begin{lstlisting}[language=ML]
        let vWith1W = [for i in 0 .. w.Length - 1 -> (float v.[i])/wf.[i]]
\end{lstlisting}

按照每个物品单位重量的价值进行降序排序,并保持与w'的对应关系。
\begin{lstlisting}[language=ML]
        let vAndvWith1W = List.zip wf vWith1W
        let vAndvWith1WSorted = List.rev (List.sortBy (fun item -> item |> snd) vAndvWith1W)
        let (w',vWith1W') = List.unzip vAndvWith1WSorted
\end{lstlisting}

将处理完成(排序好)的数据交给ksfWithSorted进行计算。
\begin{lstlisting}[language=ML]
        ksfWithSorted c' w' vWith1W'
\end{lstlisting}

完整的ksf代码如下:
\begin{lstlisting}[language=ML]
    let ksf c (w: int list) (v: int list) =
        let c' = float c
        let wf = List.map (fun item -> (item |> float)) w
        let vWith1W = [for i in 0 .. w.Length - 1 -> (float v.[i])/wf.[i]]
        let vAndvWith1W = List.zip wf vWith1W
        let vAndvWith1WSorted = List.rev (List.sortBy (fun item -> item |> snd) vAndvWith1W)

        let (w',vWith1W') = List.unzip vAndvWith1WSorted
        ksfWithSorted c' w' vWith1W'
\end{lstlisting}

\subsection{算法测试}

\begin{itemize}
\item
    测试样例:
    \begin{itemize}
    \item
        w = [50;30;45;25;5] \\
        v = [200;180;225;200;50]
    \end{itemize}
\item
    样例输出
    \begin{itemize}
    \item
        630.0
    \end{itemize}
\end{itemize}

\section{题目三}

\subsection{题目}

A simple scheduling problem.

\subsection{实现思路}

使用具有贪心性质的最短作业优先调度方案。\\
对于有任意数目的作业的情况,\(j_1\)在\(t_1\)时间结束,\(j_2\)在\(t_1+t_2\)时间结束,以此类推,平均周转时间为 \[(n*t_1+(n-1)*t_2+(n-2)*t3+...+2*t_{n-1}+t_n) \div n\]
可见\(t_1\)对平均周转时间的影响最大,\(t_2\)次之,\(t_n\)最小,故应使\(j_1\)为最短作业,\(j_n\)为最长作业。
\subsection{实现代码}

按照短作业优先策略计算平均周转时间

\begin{lstlisting}[language=ML]
    let SJF j t =
        let jobsAndTime = List.zip j t
        let jobsSortedByTime = List.sortBy (fun item -> item |> snd) jobsAndTime
        let (j',t') = List.unzip jobsSortedByTime
        let sumTime = List.mapi (fun i item -> float (item * (t'.Length - i))) t'
        (j',List.average sumTime)
\end{lstlisting}

其中:\\

\begin{lstlisting}[language=ML]
        let jobsAndTime = List.zip j t
        let jobsSortedByTime = List.sortBy (fun item -> item |> snd) jobsAndTime
        let (j',t') = List.unzip jobsSortedByTime
\end{lstlisting}
用于对作业持续时间进行升序排列,并保持与任务名称列表间的对应关系。

\begin{lstlisting}[language=ML]
        let sumTime = List.mapi (fun i item -> float (item * (t'.Length - i))) t'
        (j',List.average sumTime)
\end{lstlisting}

用于计算平均周转时间并返回计算结果和作业调度序列。

\subsection{算法测试}

\begin{itemize}
\item
    测试样例:
    \begin{itemize}
    \item
        作业序列 \ [j1;j2;j3;j4] \\
        作业持续时间 \ [15;8;3;10]
    \end{itemize}
\item
    样例输出
    \begin{itemize}
    \item
        平均周转时间最短的作业安排 \ [j3; j2; j4; j1] \\
        最短的平均周转时间 \ 17.75
    \end{itemize}
\end{itemize}

\section{题目四}

\subsection{题目}

Bin Packing

\subsection{实现思路}

使用贪心策略,从最大的物品开始装起,并尽可能将箱子装满。

\subsection{实现代码}

判断当前是否还有箱子可以装下次物品,若有,返回该箱子的索引。若没有,返回Null。\\
由于浮点数计算存在精度问题,故限定可接受的误差在0.000001内。

\begin{lstlisting}[language=ML]
            let index = Array.tryFindIndex (fun box -> (box - item) > -0.000001) boxs
\end{lstlisting}

若箱子可以装下,则装入该箱子并更新箱子剩余容量。\\
若装不下,则新放入一个箱子并装入。

\begin{lstlisting}[language=ML]
            if index.IsSome then
                boxs.[index.Value] <- boxs.[index.Value] - item
            else
                boxs <- Array.append boxs [|1.0 - item|]
\end{lstlisting}

对箱子按照剩余容量降序排序,因为物品是按降序装入的,可以让剩余容量大的箱子尽可能装满。

\begin{lstlisting}[language=ML]
            boxs <- Array.rev (Array.sort boxs)
\end{lstlisting}

完整的Packing代码如下:

\begin{lstlisting}[language=ML]
    let Packing c s =
        let s' = List.rev (List.sort s)
        let mutable boxs = Array.create 0 1.0
        List.iter (fun item -> 
            let index = Array.tryFindIndex (fun box -> (box - item) >= -0.000001 ) boxs
            if index.IsSome then
                boxs.[index.Value] <- boxs.[index.Value] - item
            else
                boxs <- Array.append boxs [|1.0 - item|]
            boxs <- Array.rev (Array.sort boxs)
            ) s'
        boxs.Length
\end{lstlisting}

\subsection{算法测试}

\begin{itemize}
\item
    测试样例:
    \begin{itemize}
    \item
        物品尺寸序列 \ [ 0.5;0.7;0.3;0.9;0.6;0.8;0.1;0.4;0.2;0.5] 
    \end{itemize}
\item
    样例输出
    \begin{itemize}
    \item
        需要的箱子数 \ 5
    \end{itemize}
\end{itemize}

\end{document}